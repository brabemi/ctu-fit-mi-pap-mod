\documentclass[12pt]{article}
\usepackage[utf8]{inputenc}
\usepackage[czech]{babel}
\usepackage{enumitem}
\usepackage{caption}
\usepackage{url}
\usepackage{graphicx}

\begin{document}

\begin{center}
\bf Semestrální projekt MI-PAP 2015/2016:\\[5mm]
    Modelování částic\\[5mm]
       Miroslav Brabenec\\
       Jan Nováček\\[2mm]
magisterské studium, FIT ČVUT, Thákurova 9, 160 00 Praha 6\\[2mm]
\today
\end{center}
%
%
%
% Co má být ve zprávě: https://edux.fit.cvut.cz/courses/MI-PAP/labs/technicka_zprava
%
%
\section{Definice problému, popis sekvenčního algoritmu a jeho implementace}

\subsection{Definice problému}
Implementujte tento algoritmus (viz \url{http://www.browndeertechnology.com/docs/BDT_OpenCL_Tutorial_NBody-rev3.html#algorithm}) a upravte ho takto:

\begin{enumerate}
\item	částice při nárazu provedou pružný odraz a
\item	každá částice má svůj náboj
\end{enumerate}

Úkol: doplňte o možnost vizualizace (alespoň exportem dat v daných časových okamžicích ve formátu vhodném pro vizualizaci v nástroji třetích stran např. ) 

\subsection{Popis sekvenčního algoritmu}
Simulátor načte nastavení simulace a počáteční umístění částic ze vstupního souboru, z toho nainicializuje hodnoty a spustí simulaci.

Simulace běží, dokud ...

\subsection{Sekvenční implementace}

\begin{center}
\begin{tabular}{c | c | c}
\textbf{Počet částic} & \textbf{Počet kroků sim.}  & \textbf{Naměřený čas} \\ \hline \hline
XX & YY & ZZ s \\ \hline
\end{tabular}
\captionof{table}{Časy sekvenčního řešení}
\end{center}


\section{Popis paralelního algoritmu a jeho implementace v OpenMP}

    Popis případných úprav algoritmu a jeho implementace, včetně volby datových struktur

    Zda byla využita vektorizace (popř. proč jí nemožno využít)

    Popis optimalizací pro dosažení lineárního zrychlení

    Tabulkově a případně graficky zpracované naměřené hodnoty časové složitosti měřených instancí běhu (optimalizované implementace) programu s popisem instancí dat

    Analýza a hodnocení vlastností dané implementace programu.

\section{Naměřené výsledky a vyhodnocení pro OpenMP}

\subsection{1. měření}
První měření bylo prováděno na vstupních datech velikosti XX částic a YY kroků simulace.

\begin{center}
\begin{tabular}{ c | c }
\textbf{Počet procesorů} & \textbf{Naměřený čas} \\ \hline \hline 
1 & ... s \\ \hline
2 & ... s \\ \hline
4 & ... s \\ \hline
8 & ... s \\ \hline
12 & ... s \\ \hline
24 & ... s \\ \hline
\end{tabular}
\captionof{table}{Časy 1. měření}
\end{center}

\begin{figure}[h]
  \begin{center}
    %~ \includegraphics[width=12cm]{images/mereni1.jpg}
    \caption{Paralelní zrychlení 1. měření} 
  \end{center}
\end{figure}

Při prvním měření jsme dosáhli ...

\end{document}
